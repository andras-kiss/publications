\documentclass[11pt,a4paper,roman]{article}
\usepackage[utf8]{inputenc}
\usepackage[scale=0.75]{geometry}
\usepackage{lastpage}
\usepackage{fancyhdr}
\rfoot{\addressfont\itshape\textcolor{gray}{\thepage\ / \pageref{LastPage}}}
\lfoot{\addressfont\itshape\textcolor{gray}{2018.10.24}}
%\chead{\addressfont\itshape\textcolor{gray}{Curriculum vit\ae~ -- András Kiss}}
\lhead{\addressfont\itshape\textcolor{gray}{Publikációs lista}}
\rhead{\addressfont\itshape\textcolor{gray}{Dr. Kiss András}}
%\pagestyle{fancy}
\date{2018. október 24.}
\title{Publikációs lista}
\author{Dr. Kiss András}
\begin{document}
\maketitle

%\begin{center}
\emph{Bírált folyóiratokban megjelent közlemények száma:} 14

\emph{Kumulatív IF:} 42.97

\emph{Hivatkozások:} 132

\emph{h-index:} 6
%\end{center}

\section{Bírált közlemények}
\begin{enumerate}

\item \textbf{András Kiss}, László Kiss, Barna Kovács, Géza Nagy, Air Gap Microcell for Scanning Electrochemical Microscopic Imaging of Carbon Dioxide Output. Model Calculation and Gas Phase SECM Measurements for Estimation of Carbon Dioxide Producing Activity of Microbial Sources, \emph{Electroanalysis 23, no. 10 (2011): 2320-2326.}, IF.: 2.14


\item Ricardo M. Souto, Javier Izquierdo, Juan José Santana, \textbf{András Kiss}, Lívia Nagy, Géza Nagy. Progress in scanning electrochemical microscopy by coupling potentiometric and amperometric measurement modes, \emph{Current Microscopy Contributions to Advances in Science and Technology, Formatex Research Center, Badajoz (2012): 1407-1415}

\item Lívia Nagy, Gergely Gyetvai, \textbf{András Kiss}, Ricardo Souto, Javier Izquierdo, Géza Nagy, Speciális célra szolgáló mikroelektródok kifejlesztése és alkalmazása, \emph{Magyar Kémiai Folyóirat 119, 2-3. (2013): 104-109.}

\item Ricardo M. Souto, \textbf{András Kiss}, Javier Izquierdo, Lívia Nagy, István Bitter, Géza Nagy, Spatially-resolved imaging of concentration distributions on corroding mag\-ne\-si\-um-based materials exposed to aqueous environments by SECM, \emph{Electrochemistry Communications 26 (2013): 25-28.}, IF.: 4.85

\item \textbf{András Kiss}, Ricardo M. Souto, Géza Nagy, Investigation of Mg/Al alloy sacrificial anode corrosion with Scanning Electrochemical Microscopy, \emph{Periodica Polytechnica Chemical Engineering 57, no. 1-2 (2013): 11-14.}, IF.: 0.30

\item Javier Izquierdo, \textbf{András Kiss}, Juan José Santana, Lívia Nagy, István Bitter, Hugh S. Isaacs, Géza Nagy, Ricardo M. Souto, Development of Mg$^{2+}$ ion-selective microelectrodes for potentiometric scanning electrochemical microscopy monitoring of galvanic corrosion processes, \emph{Journal of The Electrochemical Society 160, no. 9 (2013): C451-C459.}, IF.: 3.27

\item \textbf{András Kiss}, Géza Nagy, New SECM scanning algorithms for improved potentiometric imaging of circularly symmetric targets, \emph{Electrochimica Acta 119 (2014): 169-174.}, IF.: 4.50


\item Zsuzsanna \H{O}ri, \textbf{András Kiss}, Anton Alexandru Ciucu, Constantin Mihailciuc, Cristian Dragos Stefanescu, Lívia Nagy, Géza Nagy, Sensitivity enhancement of a ,,bananatrode'' biosensor for dopamine based on SECM studies inside its reaction layer, \emph{Sensors and Actuators B: Chemical 190 (2014): 149-156.}, IF.: 4.10

\item \textbf{András Kiss}, Géza Nagy, Deconvolution of potentiometric SECM images recorded with high scan rate, \emph{Electrochimica Acta 163 (2015): 303-309.}, IF.: 4.50

\item \textbf{András Kiss}, Géza Nagy, Deconvolution in potentiometric SECM, \emph{Electroanalysis 27, no. 3 (2015): 587-590.}, IF.: 2.14


\item A. El Jaouhari,  Dániel Filotás, \textbf{András Kiss}, M. Laabd, E. A. Bazzaoui, Lívia Nagy, Géza Nagy, A. Albourine, J. I. Martins, R. Wang, SECM investigation of electrochemically synthesized polypyrrole from aqueous medium, \emph{Journal of Applied Electrochemistry 46 (2016): 1199-1209.}, IF.: 2.223


\item Javier Izquierdo, Bibiana M Fernández-Pérez, Dániel Filotás, Zsuzsanna Őri, \textbf{András Kiss}, Romen T Martín-Gómez, Lívia Nagy, Géza Nagy, Ricardo M Souto, Imaging of Concentration Distributions and Hydrogen Evolution on Corroding Magnesium Exposed to Aqueous Environments Using Scanning Electrochemical Microscopy, \emph{Electroanalysis 28, (2016): 2354-2366.}, IF.: 2.471

\item \textbf{András Kiss}, Dániel Filotás, Ricardo M Souto, Géza Nagy, The effect of electric field on potentiometric Scanning Electrochemical Microscopic imaging, \emph{Electrochemistry Communications 77 (2017): 138-141.}, IF.: 4.569






\item \textbf{András Kiss}, Dániel Filotás, Ricardo M Souto, Géza Nagy, The effect of electric field on potentiometric Scanning Electrochemical Microscopic imaging, \emph{Electrochemistry Communications 77 (2017): 138-141.}, IF.: 4.569

\item D Filotás, BM Fernández-Pérez, J Izquierdo, \textbf{A Kiss}, L Nagy, G Nagy, RM Souto, Improved potentiometric SECM imaging of galvanic corrosion reactions, \emph{Corrosion Science 129 (2017): 136-145}, IF.: 4.245

\item D Filotás, BM Fernández-Pérez, \textbf{A Kiss}, L Nagy, G Nagy, RM Souto, Double Barrel Microelectrode Assembly to Prevent Electrical Field Effects in Potentiometric SECM Imaging of Galvanic Corrosion Processes, \emph{Journal of The Electrochemical Society. 2018 Jan 1;165(5):C270-7.}, IF.: 3.662

\end{enumerate}

\section{Konferencia előadások és poszterek}
\begin{enumerate}
\item CO$_2$ Partial Pressure Imaging in Gas Phase with Scanning Electrochemical Microscopy (SECM), poszter, \emph{X. CECE Konferencia, Pécs, 2010.}

\item Selective Amperometric Determination Of Pyrocatechol and Phenol in Wines with Flow-Injection Analysis, poszter, \emph{X. CECE Konferencia, Pécs, 2010.}

\item Four-Channel Enzyme Biosensor for Determination of Phenols in Wine, poszter, \emph{X. CECE Konferencia, Pécs, 2010.}

\item Development of a CO$_2$ microcell, and its application as measuring tip in Scanning Electrochemical Microscope. Scanning in gas phase over biological samples, előadás, \emph{XXXIV. Szegedi Kémiai Előadói Napok, Szeged, 2011.}

\item Investigation of Mg/Al alloy sacrificial anode corrosion with Scanning Electrochemical Microscopy, poszter, \emph{Műszaki Kémiai Napok 2012, Veszprém, 2012.}

\item Investigation of galvanic corrosion of the Fe-Mg galvanic pair with Scanning Electrochemical Microscope, poszter, \emph{Kémiai Szenzorok Workshop 2012, Pécs, 2012.}

\item Fabrication of a new, solid contact Mg$^{2+}$ ion-selective electrode, and its application in Scanning Electrochemical Microscopic corrosion studies, előadás, \emph{1. Interdiszciplináris Doktorandusz Konferencia, Pécs, 2012.}

\item A new, solid contact Mg$^{2+}$ ion-selective electrode as measuring tip for Scanning Electrochemical Microscope in corrosion studies, előadás, \emph{Szentágothai János Emlékkonferencia, Pécs, 2012. október 29-30.}

\item New insights in the corrosion mechanism of magnesium by SECM, előadás, \emph{7th Workshop on Scanning Electrochemical Microscopy (SECM) and Related Techniques, Ein Gedi, Izrael, 2013. február 17-21.}

\item High-speed potentiometric SECM imaging of radially symmetric targets, előadás, \emph{ESEAC Malmö, Svédország, 2013. június 11-14.}

\item Deconvolution of potentiometric SECM images recorded with high scanrate, poszter, \emph{Mátrafüred Konferencia, 2014. június 13-16., Visegrád, Magyarország.}

\item High-speed SECM imaging, plenáris előadás, \emph{Analytica Conference 2016. május 10-13., München, Németország.}

\item The effect of electric field on potentiometric Scanning Electrochemical Microscopic imaging, Poster presentation, \emph{Mátrafüred Conference 2017 11-16 június, Visegrád, Hungary.}

\item High-speed SECM imaging, Poster presentation, \emph{9th Workshop on Scanning Electrochemical Microscopy and Related Techniques, 2017 13-17 augusztus, Varsó, Lenygelország.}

\item Mapping the Belousov--Zhabotinsky oscillating reaction with the Scanning Electrochemical Microscope, \emph{Analitika Napok, 2018 23-24 április, Balatonszemes, Magyarország.}

\item Potentiometric scanning electrochemical microscopic mapping of the distributed Belousov-Zhabotinsky oscillating reaction, \emph{1st International Conference on Reaction Kinetics, Mechanisms and Catalysis, RKMC 2018, Budapest, Magyarország}


\end{enumerate}

\end{document}
